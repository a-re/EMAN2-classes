\section{Testing Framework}

EMAN2 uses python unit test and regression test as its testing
framework. For features not testable or not easily to be tested in
python, they will be tested in C++.


\subsection{Testing Guidelines}
    \begin{itemize}
      \item
        Each unit test must be self contained. 
      \item
	Each test module (.py file) should  be named as "test\_" +
	featurename.
      \item
	Each test method should be named as "test\_" + method-name.
      \item
	unittest module should be used to do the unit testing. doctest
	is discouraged.
    \end{itemize}

\subsection{Test Example}
\begin{itemize}

  \item
    Example of a unit test and a regression test:

    \begin{verbatim}import unittest
from test import test_support

class MyTestCase1(unittest.TestCase):

    # Only use setUp() and tearDown() if necessary

    def setUp(self):
        ... code to execute in preparation for tests ...

    def tearDown(self):
        ... code to execute to clean up after tests ...

    def test_feature_one(self):
        # Test feature one.
        ... testing code ...

    def test_feature_two(self):
        # Test feature two.
        ... testing code ...

    ... more test methods ...

class MyTestCase2(unittest.TestCase):
    ... same structure as MyTestCase1 ...

... more test classes ...

def test\_main():
    test\_support.run\_unittest(MyTestCase1,
                              MyTestCase2,
                              ... list other tests ...
                             )

if __name__ == '__main__':
    test_main()
    \end{verbatim}
    \normalcolor

  \item Here is a more detailed example:    

    \begin{verbatim}import random
import unittest

class TestSequenceFunctions(unittest.TestCase):
    
    def setUp(self):
        self.seq = range(10)

    def tearDown(self):
	# do clean up here

    def testshuffle(self):
        # make sure the shuffled sequence does not lose any elements
        random.shuffle(self.seq)
        self.seq.sort()
        self.assertEqual(self.seq, range(10))

    def testchoice(self):
        element = random.choice(self.seq)
        self.assert_(element in self.seq)

    def testsample(self):
        self.assertRaises(ValueError, random.sample, self.seq, 20)
        for element in random.sample(self.seq, 5):
            self.assert_(element in self.seq)

if __name__ == '__main__':
    unittest.main()\end{verbatim}
    \normalcolor

  \item
    Asserting Values and Conditions
    \begin{itemize}
      \item[]
	The crux of each test is a call to assertEqual() to check for
	an expected result; assert\_() to verify a condition; or
	assertRaises() to verify that an expected exception gets
	raised.  
    \end{itemize}

\end{itemize}
