%\documentclass{manual}
%\usepackage{url, hyperref}
%\documentclass{article}
%\usepackage{hyperref}
%\begin{document}

\section{Required Libraries / Programs}
\label{REQUIRED-LIBRARIESS} \index{Installation!Required Libraries}
The following libraries are required for EMAN2 installation (the
libraries should be installed as shared-object libraries where applicable):

\subsection{FFTW} 
  Version 2.1.3+\\
  Available at: \href{http://www.fftw.org/}{http://www.fftw.org/} 
   \index{Installation!Required Libraries!FFTW} \index{FFTW}

   To install fftw from source use either configure option:

   \% ./configure --enable-static=no --enable-shared=yes --enable-float --enable-type-prefix

   OR

   \% ./configure --enable-shared=yes --enable-float 
   
   
   Followed by:
   
   \% make

\subsection{GSL - GNU Scientific Library} 
  Version: 1.3+\\
  Available at: \href{http://www.gnu.org/software/gsl/}{http://www.gnu.org/software/gsl/} 
   
  Installation is very straight forward:

   \% ./configure
   
   \% make
   \normalcolor

\subsection{Boost}
  Version: 1.32 or lower\\
  Available at: \href{http://www.boost.org}{http://www.boost.org}
  \index{Installation!Required Libraries!Boost} \index{Boost}
       
  (NOTE:  EMAN2 does not currently support Boost versions above 1.32.)  

  \begin{enumerate}
    \item
      Installing Boost requires Boost.Jam.  Executables and
      source code for jam can be found at the Boost website.
  \end {enumerate}

  Installing Boost requires the user to identify a particular
  toolset to use during compliation.  Most UNIX systems will
  probably use the "gcc" toolset; visit
  \href{http://www.boost.org/more/getting\_started.html\#Tools}{http://www.boost.org/more/getting\_started.html\#Tools}
  for a complete listing.        

  \% bjam "-sTOOLS=gcc" install

  Header files from the Boost installation (located in the
  "boost" subdirectory of the Boost installation
  (ex. /boost\_1\_32\_0/boost)) must now either be added to the
  compilers path or copied into an existing location on the path
  in a subdirectory /boost.

  One possibility for this might be:
  \% cp -r boost /usr/include/boost
      
  
\subsection{CMake}
  Version: 2.0.6+\\
  Available at: \href{http://www.cmake.org}{http://www.cmake.org}
   \index{Installation!Required Libraries!CMake} \index{CMake}

       Executables for several platforms are available; source code
       can also be used for custom installations.


\section{Optional Libraries / Programs} 
\label{OPTIONAL-LIBRARIES} \index{Installation!Optional Libraries}

  - To read/write HDF5 image, use hdf5 (\href{http://hdf.ncsa.uiuc.edu/HDF5}{http://hdf.ncsa.uiuc.edu/HDF5}).
	  (NOTE: HDF5 1.6.4 has some API compatibility issue and it doesn't work	        with EMAN2 yet.)
	  
    - To read TIFF image, use libtiff (\href{http://www.libtiff.org}{http://www.libtiff.org})

    - To read PNG image, use PNG (\href{http://www.libpng.org}{http://www.libtiff.org})

    For development the following libraries/programs are required (see
    \ref{HELP-OPT-PROG-INST} for installation help):

    - Python (version 2.2+)     (\href{http://www.python.org}{http://www.python.org})
    
    - Boost Python (version 1.32-)	(\href{http://www.boost.org}{http://www.boost.org})
    
    - Numerical Python Numpy (version 22.0+)
                                        (\href{http://www.pfdubois.com/numpy}{http://www.pfdubois.com/numpy})



\section{Quick Installation} 
\label{QUICK-INSTALLATION} \index{Installation!Quick Installation}

 Suppose you have source code eman2.tar.gz

\begin{enumerate}
    \item
      \begin{itemize}
          \item[\%] cd \$HOME
          \item[\%] mkdir -p EMAN2/src
          \item[\%] cd EMAN2/src
          \item[\%] gunzip eman2.tar.gz
          \item[\%] tar xf eman2.tar \\
      \end{itemize}
      \normalcolor
    \item
      \begin{itemize}
	  \item[\%]mkdir build
	  \item[\%]cd build \\
       \end{itemize}
      \normalcolor

     \item       
       \begin{itemize}
	 \item[\%]cmake ../eman2
         \item[\%]make
         \item[\%]make install \\
       \end{itemize}
       \normalcolor

     \item
          set up login shell\\
          for csh/tcsh, put the following to your .cshrc or .tcshrc file:
	  
          setenv EMAN2DIR \$HOME/EMAN2\\
          setenv PATH \$EMAN2DIR/bin:\$\{PATH\}\\
          setenv LD\_LIBRARY\_PATH  \$EMAN2DIR/lib\\
          setenv PYTHONPATH .:\$HOME/EMAN2/lib\\
	  \normalcolor

	  for bash in .bashrc add:
	
          export EMAN2DIR=\$HOME/EMAN2\\
          export PATH=\$PATH:\$EMAN2DIR/bin\\
          export LD\_LIBRARY\_PATH=\$EMAN2DIR/lib\\
          export PYTHONPATH=\$PYTHONPATH:\$HOME/EMAN2/lib
	  \normalcolor
\end{enumerate}


\section{Advanced Installation} 
\label{ADVANCED-INSTALLATION} \index{Installation!Advanced Installation}

  If your libraries (fftw, gsl, hdf, etc) are not found by Quick
   Installation, or if you want to change the compilation options,
   the following steps help:

   \begin{enumerate}
     \item
       follow the first 2 steps in Quick Installation.
       
     \item
        If your libraries are not installed at the default places,
          setup the related environment variables:
	  \begin{itemize}
            \item[-] fftw  \(\longrightarrow\) FFTWDIR
            \item[-] gsl  \(\longrightarrow\) GSLDIR
            \item[-] tiff \(\longrightarrow\) TIFFDIR
            \item[-] png  \(\longrightarrow\) PNGDIR
            \item[-] hdf5 \(\longrightarrow\) HDF5DIR
            \item[-] python \(\longrightarrow\) PYTHON\_ROOT and PYTHON\_VERSION
          \end{itemize}
	
	\item
        \% ccmake ../eman2
	\begin{itemize}
          \item[-] type 'c' if it asks about "CMAKE\_BACKWARDS\_COMPATIBILITY".
          \item[-] make necessary changes for compilation flags.
	    \begin{itemize}
	    \item[-]developers will probably want to set BOOST-LIBRARY to a
              Boost.Python object file
              (ex. libboost\_python-gcc-1\_32.so)
	    \end{itemize}
          \item[-] Then type 'c', and type 'g'.
	\end{itemize}

	\item
	  \begin{itemize}
          \item[\%] make
          \item[\%] make install          
	  \end{itemize}
	  \normalcolor
   \end{enumerate}
                
\subsection{Platform Dependent Optimization}


    In CMake Configuration, enable the following option for your platform:
    \begin{itemize}
        \item[-] Athlon:  ENABLE\_ATHLON
        \item[-] Opteron: ENABLE\_OPTERON
        \item[-] Mac G5:  ENABLE\_G5
    \end{itemize}
          
\subsection{Generating the  Latest Documentation}

  \begin{enumerate}
    \item
      Install doxygen (version 1.4.3+, \href{http://www.doxygen.org}{http://www.doxygen.org})

    \item
      Run ``sh ./makedoc.sh'' from the ``EMAN2/src/eman2/'' directory to
   generate the documentation in the ``EMAN2/src/eman2/doc/''.
  \end{enumerate}


\section{Notes for Developers} 
\label{DEVELOPERS-NOTES} \index{Developer's Guide}

\begin{enumerate}
  \item
    For Emacs users, please add the following line to your \$HOME/.emacs:
       (setq default-tab-width 4)

  \item
    Ensure Boost.Python is installed

  \item
    EMAN2 uses Pyste (\href{http://www.boost.org/libs/python/pyste/}{http://www.boost.org/libs/python/pyste/}) to wrap C++
    into python. Here is the way to install Pyste:

    \begin{enumerate}
      \item
        get boost python.\\
        \% cd libs/python/pyste/install \\
        \% python setup.py install \\\normalcolor
	\item
          install elementtree
	\item
          install GCCXML
	\item
         for boost 1.32.0, apply a patch for PYSTE.
           (Contact EMAN2 developers for the patch.)
    \end{enumerate}
    
    \item
      To generate new boost python wrapper, run
      \begin{itemize}
        \item[\%] cd eman2/libpyEm
        \item[\%] ./create\_boost\_python
      \end{itemize}
      \normalcolor

      \item
	Windows Installer
	EMAN uses "Nullsoft Scriptable Install System" (\href{http://nsis.sourceforge.net/}{http://nsis.sourceforge.net/})
	to generate the windows installer.
	It also uses "HM NIS Edit" (\href{http://hmne.sourceforge.net/}{http://hmne.sourceforge.net/}) as the editor.
	\end{enumerate}



%\end{document}